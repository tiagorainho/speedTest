\documentclass{report}
\usepackage[T1]{fontenc} % Fontes T1
\usepackage[utf8]{inputenc} % Input UTF8
\usepackage[backend=biber, style=ieee]{biblatex} % para usar bibliografia
\usepackage{csquotes}
\usepackage[portuguese]{babel} %Usar língua portuguesa
\usepackage{blindtext} % Gerar texto automaticamente
\usepackage[printonlyused]{acronym}
\usepackage{hyperref} % para autoref
\usepackage{graphicx}
\usepackage{amsmath}

\bibliography{bibliografia}


\begin{document}
%%
% Definições
%
\def\titulo{Trabalho de Aprofundamento 2: Speedtesting}
\def\data{31/03/2019}
\def\autores{João Tiago Rainho, Sofia Teixeira Vaz}
\def\autorescontactos{(92984) tiago.rainho@ua.pt, (92968) sofiateixeiravaz@ua.pt}
\def\versao{VERSAO}
\def\departamento{DEPARTAMENTO DE ELETRÓNICA, TELECOMUNICAÇÕES E INFORMÁTICA}
\def\empresa{UNIVERSIDADE DE AVEIRO}
\def\logotipo{ua.pdf}
%
%%%%%% CAPA %%%%%%
%
\begin{titlepage}

\begin{center}
%
\vspace*{50mm}
%
{\Huge \titulo}\\ 
%
\vspace{10mm}
%
{\Large \empresa}\\
%
\vspace{10mm}
%
{\LARGE \autores}\\ 
%
\vspace{30mm}
%
\begin{figure}[h]
\center
\includegraphics{\logotipo}
\end{figure}
%
\vspace{30mm}
\end{center}
%
\begin{flushright}
\versao
\end{flushright}
\end{titlepage}

%%  Página de Título %%
\title{%
{\Huge\textbf{\titulo}}\\
{\Large \departamento\\ \empresa}
}
%
\author{%
    \autores \\
    \autorescontactos
}
%
\date{\data}
%
\maketitle

\pagenumbering{roman}

%%%%%% RESUMO %%%%%%
\begin{abstract}
Este trabalho, que se baseia na criação de um cliente para fazer \textit{speedtesting}, visa a proteção do utilizador, como foi referido na introdução. No entanto, a nossa ferramenta, apesar de eficiente, não é o melhor disponível, uma vez que diversas empresas já fizeram ferramentas parecidas com resultados mais viáveis.
\end{abstract}

%%%%%% Agradecimentos %%%%%%
% Segundo glisc deveria aparecer após conclusão...
%\renewcommand{\abstractname}{Agradecimentos}
%\begin{abstract}
%Eventuais agradecimentos.
%Comentar bloco caso não existam agradecimentos a fazer.
%\end{abstract}


\tableofcontents
\listoftables     % descomentar se necessário
% \listoffigures    % descomentar se necessário


%%%%%%%%%%%%%%%%%%%%%%%%%%%%%%%
\clearpage
\pagenumbering{arabic}

%%%%%%%%%%%%%%%%%%%%%%%%%%%%%%%%
\chapter{Introdução}
\label{chap.introducao}
A internet, apesar de ser vendida como algo tangível, não o é. Assim, o consumidor pode facilmente ser enganado pela operadora, uma vez que pode estar a pagar por uma qualidade que não está a receber e nem se aperceber disso. É nesse contexto que a nossa ferramenta de \textit{speedtesting} foi criada, no entanto, há mais razões. Uma destas pode ser verificação de tráfego, e outra ainda é verificação da qualidade da placa de rede do computador em si.
Este documento está dividido em quatro capítulos.
Depois desta introdução,
no \autoref{chap.metodologia} é apresentada a metodologia seguida,
no \autoref{chap.resultados} são apresentados os resultados obtidos,
sendo estes discutidos no \autoref{chap.analise}.
Finalmente, no \autoref{chap.conclusao} são apresentadas
as conclusões do trabalho.

\chapter{Metodologia}
\label{chap.metodologia}
Na fase inicial, tentámos perceber como dividir melhor o trabalho para se conseguir fazer o código da maneira mais simples possível. Para isto, usámos os pontos especificados no guião, trocando a ordem de alguns. 
Ainda adicionámos algumas features que não são mencionadas no guião. Por exemplo, quando o nome do país inserido não existe no ficheiro de servidores, ligamo-nos a um servidor qualquer da lista.

\section{Cálculo da latência}
Para calcular a latência, calculámos o tempo que o server demora a responder PONG após receber PING.
Neste cálculo, decidimos ignorar o primeiro PING-PONG, uma vez que esse valor seria sempre muito maior do que os outros. Assim, em vez de corrermos o ciclo 11 vezes (para calcular 10 latências), corremos o ciclo 12 vezes.

\section{Cálculo da largura de banda}
Decidimos que a melhor maneira de calcular a largura de banda seria receber 1 MB de informação (para abrir os portos de informação), enviar a restante informação (INFO - 1 MB) e ver quanto tempo isso demorou (t). Assim, o cálculo seria
\[\frac{\text{INFO - 1 MB}}{\text{t}}\]

\section{Modulação do código}
Após fazermos o código necessário para o cálculo da largura de banda, apercebemo-nos de um problema - o nosso código não era modulável de maneira nenhuma. Assim, decidimos pegar no código que tínhamos e fazer com que cada parte fosse uma função. Assim, cada função poderia ser usada sozinha num outro programa qualquer. É de frizar que fazer um programa altamente modulável é uma boa prática de programação e, além disso, é mais fácil testar no que toca à procura de \textit{bugs} e resolução desses mesmos.

\section{Escrita dos documentos}
Para a escrita, decidimos fazer tudo da maneira mais simples, invocando as funções de cálculo de latência e de cálculo de largura de banda dentro da função que escreve o ficheiro em si. Decidimos abrir o ficheiro em modo "a" em vez de "w", uma vez que, ao abrir em "w", o ficheiro é apagado, enquanto que, abrindo em "a", o texto é adicionado (\textit{appended}).
Mais tarde, apercebemo-nos que, antes de escrever os dados, temos de limpar o ficheiro \textit{report.csv} de modo a que o ficheiro tenha apenas os dados do último teste que foi feito. Assim, antes de iniciar a escrita do ficheiro, abrimo-lo em modo "wb", o que faz com que o conteúdo do ficheiro seja apagado antes de se começar a escrever.

\section{Robustez}
No que toca às exceções, decidimos verificar apenas se ocorreu \textit{timeout} ou se houve um \textit{socket error}. Assim, sempre que enviamos ou recebemos informação, testamos estas duas exceções. 
No que toca à existência do ficheiro \textit{servers.json}, decidimos que, quando ocorresse um erro, o programa iria fechar. Esta é a única rota que faz sentido, uma vez que, se o ficheiro não existir, não será possível entrar em contacto com nenhum server e, assim, não faz sentido fazer o teste.

\section{Encriptação}
Para a cifragem, decidimos, após a escrita do \textit{report.csv}, reler esse ficheiro e cifrá-lo com a key recebida no ficheiro com a chave a usar (\textit{key.priv}). Depois fazemos o hash desse valor e guardamo-lo no ficheiro \textit{report.sig}.


\chapter{Análise}
\label{chap.analise}
O ping deu valores muito semelhantes, no entanto a largura de banda apresentou alguns desvios dependendo da internet a utilizar, em casa os valores estavam muitissimo parecidos, no entanto em cafés apresenta resultados maior descrepância de valores. 

\chapter{Conclusões}
\label{chap.conclusao}
O projeto deu resultados muitissimo aproximados dos resultados reais recebidos por sites destinados a este tipo de testes, logo consideramos que fomos bem sucedidos neste projeto.

\chapter*{Contribuições dos autores}
Ambos os estudantes deste projeto participaram aproximadamente o mesmo.

%%%%%%%%%%%%%%%%%%%%%%%%%%%%%%%%%
\chapter*{Acrónimos}
\begin{acronym}
\acro{ua}[UA]{Universidade de Aveiro}
\acro{miect}[MIECT]{Mestrado Integrado em Engenharia de Computadores e Telemática}
\acro{lei}[LEI]{Licenciatura em Engenharia Informática}
\acro{glisc}[GLISC]{Grey Literature International Steering Committee}
\end{acronym}


%%%%%%%%%%%%%%%%%%%%%%%%%%%%%%%%%
\printbibliography

\end{document}
